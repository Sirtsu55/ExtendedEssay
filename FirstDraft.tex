\documentclass[12pt]{article}


% Preamble
\setlength{\parindent}{0pt}

% Document Information
\title{Optimal Method of Ray Tracing Voxels with Hardware Acceleration in Real-Time}
\author{Srijan Dhungana}
\date{2024}



% Begin Document
\begin{document}

% Title Page
\maketitle
\clearpage

% Table of Contents
\tableofcontents
\clearpage

\section{Motivation}

My first introduction to ray tracing was when NVIDIA announced their RTX series GPUs in 2018.
The idea of simulating light rays in real-time was unreal to me and the results were stunning.
I wanted to make my own hardware ray tracer and so I was introduced to computer graphics.
I have always been fascinated by the aesthetics of voxels based games and I wanted to combine the two.
However, when researching, I found that there was no information on the fastest method to ray trace a voxel scene.
This essay is an attempt to answer that question.

% Introduction
\section{Introduction}

\subsection{Ray Tracing}

Ray tracing is a method to calculate where a given ray (line), intersects
with the environment. This is useful to simulate light rays. It is used in film
production, video games, optics, medical imaging, architectural visualization,
and many other fields. In physics, light rays bounce in the environment, thus
ray tracing is used to calculate the position of intersection where a light ray
hits the environment. Then the programmer is free to repeat the process
and simulate the ray bouncing off into other directions. This process is a
highly parallelizable process, thus it is perfectly suited for GPUs which have
thousands of cores. However, due to how computationally expensive the
process is, it is not feasible to do in real-time without dedicated hardware
acceleration. Due to those limitations, it was only used in non-real-time
applications such as film production and pre-rendered graphics. Hardware
acceleration ray tracing was introduced in 2018 by Nvidia. Since then, ray
tracing has been possible in real-time and further supported by AMD and
Intel GPUs. This has allowed real-time applications to leverage ray-tracing
for realistic lighting, reflections, and shadows.

\subsection{Voxels}

Voxels are a way to represent 3D data. It is similar to pixels in 2D images,
but in 3D space. Essentially, voxels are cubes that are located in 3D space.
They are used in medical imaging, simulations, and video games; they have
similar applications to ray tracing. Voxels are convenient, because they rep-
resent volumetric data, which is difficult to represent with polygons. For
example, a physics simulation of a fluid would be difficult to represent with
polygons, but easy with voxels. Each voxel can store data such as color,
density, temperature, etc. at each point in 3D space. The resolution of the
simulation can also be adjusted easily just by changing the size of the voxels.

\subsection{Research Question}

The research question is: What is the difference in performance between
purely hardware based voxel ray tracing and a mixed software-hardware ap-
proach?

\section{Background Information}
\subsection{Mathematics of Ray Tracing}
\subsection{GPU Architecture}

\section{Methods}
\subsection{Hardware BVH Traversal}
\subsection{Signed Distance Fields}
\subsection{Other Algorithms}

\section{Memory Consumption}

\section{Results}

\section{Conclusion}


\end{document}