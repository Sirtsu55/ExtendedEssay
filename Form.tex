\documentclass[12pt]{article}


% Preamble
\setlength{\parindent}{0pt}

% Document Information
\title{Optimal Method of Ray Tracing Voxels with Hardware Acceleration in Real-Time}
\author{Srijan Dhungana}
\date{2024}



% Begin Document
\begin{document}

% Title Page
\maketitle
\clearpage

% Table of Contents
\tableofcontents
\clearpage

% Introduction
\section{Introduction}

Ray tracing is a method to calculate where a given ray (line), intersects with the environment.
This is useful to simulate light rays.
It is used in film production, video games, optics, medical imaging, architectural visualization, and many other fields.
Light rays bounce in the environment, thus ray tracing is used to calculate the position of intersection where a light ray hits the environment. 
Then the programmer is free to repeat the process and simulate the ray bouncing off into other directions.
This process is a highly parallelizable process, thus it is perfectly suited for GPUs which have thousands of cores.
However, due to how computationally expensive the process is, it is not feasible to do in real-time without dedicated hardware acceleration.
Due to those limitations, it was only used in non-real-time applications such as film production and pre-rendered graphics.

\bigskip

Hardware acceleration ray tracing was introduced in 2018 by Nvidia.
Since then, ray tracing has been possible in real-time and supported by AMD and Intel GPUs.
This has allowed real-time applications to leverage ray-tracing for realistic lighting, reflections, and shadows.



\section{Background Information}
\subsection{Mathematics of Ray Tracing}
\subsection{GPU Architecture}

\section{Methods}
\subsection{Hardware BVH Traversal}
\subsection{Signed Distance Fields}
\subsection{Other Algorithms}

\section{Memory Consumption}

\section{Results}

\section{Conclusion}


\end{document}